\chapter{Objetivos}
\label{cap:capitulo2}

En este capítulo se detallan los objetivos del trabajo realizado, así como, los requisitos que este ha de cumplir, la metodología utilizada y el plan de trabajo seguido para completarlo.\\

\section{Descripción del problema}
\label{sec:descripcion}

El objetivo del trabajo es implementar un coche autónomo bajo una plataforma de bajo coste y reducido tamaño capaz de circular por un circuito o carretera en un entorno dinámico interactuando con objetos propios de una ciudad, como semáforos, señales de stop o peatones. Para ello se ha dividido el objetivo en general en dos subobjetivos:

\begin{enumerate}[(a)]
	\item Entorno simulado: utilizando el entorno de simulación \textit{Gazebo} se desarrollará el problema anteriormente descrito con la finalidad de, posteriormente, realizar lo mismo en un entorno real.\\
	\item Entorno real: utilizando un robot real diseñado a partir de la placa \textit{NVIDIA Jetson Nano}, se desarrollará el problema anteriormente descrito sobre un circuito construido a partir de pistas de \textit{Scalextric}\footnote{\url{https://scalextric.es/}}.\\
\end{enumerate}\

Como se dijo anteriormente, en ambos entornos, se ha de completar dos objetivos:

\begin{enumerate}[(a)]
	\item Seguimiento de carril: el coche autónomo tendrá que ser capaz de realizar un seguimiento del carril utilizando una red neuronal que indicará el centro del carril al que el robot deberá ceñirse.
	\item Detección de objetos: mientras el robot realiza el seguimiento del carril debe interactuar con objetos del entorno, esto es, reaccionar cuando ve una señal de stop o un semáforo y rojo y detenerse apropiadamente. Está detección deberá ser realizada, preferiblemente, en tiempo real.
\end{enumerate}\

\section{Requisitos}
\label{sec:requisitos}
El trabajo tendrá una serie de requisitos que deberán ser respetados:
\begin{enumerate}
	\item El sistema operativo utilizado, para ambos entornos, será \textit{GNU/Linux}, concretamente la distribución \textit{Ubuntu} 20.04 LTS ya que cuenta con soporte para múltiples arquitecturas y proporciona un gran rendimiento.
	\item El entorno simulado requerirá la presencia de una tarjeta gráfica dedicada ya que es muy recomendable para trabajar con redes neuronales; concretamente de la marca \textit{NVIDIA}, ya que se utilizará la plataforma \textit{CUDA}.
	\item El entorno real requerirá un robot con la placa de desarrollo \textit{NVIDIA Jetson Nano} debido a ser una de las placas con \textit{GPU} más económicas.
	\item El lenguaje de programación utilizado será \textit{Python} debido a las librerías utilizadas, que se detallarán en el próximo capítulo.
\end{enumerate}\

\section{Metodología}
\label{sec:metodologia}
Partiendo de los requisitos y objetivos previamente descritos, se procedió a evaluar el hardware necesario; a continuación, se realizó un análisis de diversas bibliotecas de código con el objetivo de seleccionar las que fuesen compatibles y tuviesen un mejor rendimiento en la plataforma de hardware elegida. El siguiente paso fue el diseño del software necesario y cómo integrarlo con las bibliotecas escogidas. Por último se realizaron pruebas periódicas tanto en simulador como en un entorno real, con el objetivo de ir afinando el software para conseguir el resultado final.\\

\section{Plan de trabajo}
\label{sec:plantrabajo}

El plan de trabajo se ha basado en reuniones semanales o quincenales con el tutor, dependiendo de la carga de trabajo, en las que se iban fijando objetivos específicos y fijando la estrategia para poder completar el proyecto.\\

Todo el trabajo realizado se ha ido subiendo a un repositorio de trabajo en \textit{GitHub}\footnote{\url{https://github.com/jmvega/tfg-amariscal}}. De esta forma se podía analizar de forma muy rápida los cambios realizados en el código. También se ha ido desarrollando una \textit{Wiki}\footnote{\url{https://github.com/jmvega/tfg-amariscal/wiki}} en \textit{GitHub} a modo de bitácora en la que se iban detallando los avances del proyecto, así como, los problemas y las limitaciones que iban surgiendo a medida que se cumplían los objetivos.\\
