\cleardoublepage

\chapter*{Resumen\markboth{Resumen}{Resumen}}
El problema abordado se enmarca dentro del ámbito de la robótica y la visión artificial. Se trata de dos campos en auge actualmente que proporcionan soluciones eficaces en multitud de campos de aplicación, como por ejemplo la conducción autónoma donde un vehículo sería capaz de circular sin conductor. De esta forma trayectos largos como los realizados en el transporte de mercancías serían llevados a cabo en un menor tiempo y con una mayor seguridad.\\

El objetivo de este proyecto es desarrollar un coche autónomo sobre una plataforma de bajo coste y reducido tamaño capaz de circular por un circuito o carretera en un entorno dinámico interactuando con objetos propios de una ciudad. El objetivo propuesto se ha desarrollado en dos entornos distintos; en un entorno simulado, donde se realizan diversas pruebas con el objetivo de comprobar la viabilidad de la solución planteada para, a continuación, reproducir ese mismo escenario en un entorno real, implementando la solución sobre un robot real.\\

El objetivo planteado se ha resuelto a través del uso de dos redes neuronales, una para el seguimiento de carril, que se combina con un controlador que utiliza la salida de dicha red para comandar una determinada velocidad lineal y angular al robot. Y otra red neuronal que tiene como objetivo detectar los objetos presentes en el entorno y reaccionar en consecuencia a estos.\\

Al tratarse de una plataforma de bajo coste, se han encontrado limitaciones propias de la potencia que ofrece el equipo que compone el robot. A pesar de estas dificultades, se ha conseguido resolver el problema propuesto, principalmente reduciendo la resolución de la imagen que reciben las redes neuronales y realizando una optimización del equipo. Además, se han planteado otros ámbitos de aplicación distintos a la conducción autónoma en los que el software desarrollado podría ser de utilidad.\\
