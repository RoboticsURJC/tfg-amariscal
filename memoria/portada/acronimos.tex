\cleardoublepage

\chapter*{Acrónimos\markboth{Acrónimos}{Acrónimos}}

% Añade a continuación los acrónimos que uses en el documento. Algunos ejemplos:
\begin{acronym}
	\acro{AI}{\emph{Artificial Intelligence}}
	\acro{ANN}{\emph{Artificial Neural Network}}
	\acro{AGV}{\emph{Automated Guided Vehicle}}
	\acro{AMR}{\emph{Autonomous Mobile Robot}}
	\acro{CPU}{\emph{Central Processing Unit}}
	\acro{GPU}{\emph{Graphics Processing Unit}}
	\acro{EO/IR}{\emph{Electro-Optical and Infrared Sensors}}
	\acro{FIR}{\emph{Far Infrared}}
	\acro{GPIO}{\emph{General Purpose Input/Output}}
	\acro{RPM}{\emph{Revolutions Per Minute}}
	\acro{PWM}{\emph{Pulse With Modulation}}
	\acro{CW}{\emph{Clockwise}}
	\acro{CCW}{\emph{Counter Clockwise}}
	\acro{LIPO}{\emph{Lithium-Ion Polymer Battery}}
	\acro{USB}{\emph{Universal Serial Bus}}
	\acro{YOLO}{\emph{You Only Look Once}}
	\acro{CNN}{\emph{Convolutional Neural Network}}
	\acro{GUI}{\emph{Graphical User Interface}}
	\acro{MSE}{\emph{Mean Squared Error}}
	\acro{FPS}{\emph{Frames Per Second}}
	\acro{HSV}{\emph{Hue Saturation Value}} 
	\acro{RGB}{\emph{Red Green Blue}}
	\acro{RGBD}{\emph{Red Green Blue Depth}}
	\acro{FP16}{\emph{16-bit Floating Point}}
	\acro{FP32}{\emph{32-bit Floating Point}}
\end{acronym}
